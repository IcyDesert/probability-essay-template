\documentclass{article}
\usepackage{amsmath}
\usepackage[UTF8]{ctex}
\usepackage{graphicx}
\usepackage{fontspec}
\usepackage{array}
\usepackage{setspace} % 扩展表格功能
\usepackage{enumitem} % 用于自定义列表样式
\usepackage{titlesec} % 用于自定义标题格式
\usepackage[a4paper, margin=2.2cm, left=2.5cm]{geometry}
\usepackage[style=gb7714-2015, citestyle=numeric, backend=bibtex8]{biblatex}


\zihao{-4}
\setstretch{1.25}
\setCJKmainfont{simsun.ttc}[AutoFakeBold]
\setmainfont{Times New Roman}

% 图片放在 figure 文件夹里 
\graphicspath{{./figure/}}
\DeclareGraphicsExtensions{.pdf,.jpeg,.png,.jpg}

% 参考文献文件名
\addbibresource{db.bib}
\defbibheading{bibliography}[\refname]{
    \centering % 居中
    \zihao{-2}\heiti
    \textbf{参考文献} % 设置标题
    \vspace{1em} % 设置标题与内容之间的间距
}

% 一些自定义的指令,方便用于数学公式
\newcommand{\dx}{\mathrm{d}x} % dx
\newcommand{\dy}{\mathrm{d}y} % dy
\newcommand{\dif}[1]{\mathrm{d}#1} % 对任意字母进行微分,如 \dif{z}
\newcommand{\abso}[1]{\lvert#1\rvert} % 绝对值符号,如\abso{x}

\begin{document}
% 封面页信息
{\centering\zihao{2}
    \textbf{概率论与数理统计课程小论文} \\
    \vspace*{1em}
    {\heiti 这是题目} \\
}
\vspace*{8em}
\begin{center}
    {\bfseries\zihao{-2}
        \begin{tabular}{|>{\centering\arraybackslash}m{3cm}|c|}
            \hline
            班 \quad 号  & 23******   \\
            \hline
            学 \quad 号 & 23********** \\
            \hline
            姓 \quad 名 & 某某某        \\
            \hline
            日 \quad 期 & 2024.12.** \\
            \hline
            得 \quad 分 &            \\
            \hline
        \end{tabular}
    }
\end{center}
\newpage
% 摘要及关键词
{\centering
    \bfseries\zihao{-2}\heiti 摘\quad 要\par
    % 神秘,不加 \par 无法居中
}

摘要是论文内容的高度概括,应具有独立性和自含性,即不阅读论文的全文,就能获得必要的信息。摘要应包括本论文的目的、主要研究内容、研究方法、创造性成果及其理论与实际意义。摘要中不宜使用公式、化学结构式、图表和非公知公用的符号与术语,不标注引用文献编号,同时避免将摘要写成目录式的内容介绍。\\ 

关键词:关键词1;关键词2;…… \\

% 标题
{
\zihao{3}\noindent\textbf{课题背景及研究的目的和意义} \\
}

1828年,R.R.Willis发表了一篇关于小孔节流平板中压力分布的文章,这是有记载的研究气体润滑的最早文献。

这是参考文献写法\cite{fenbuhanshu}

\newpage
\printbibliography
\vfill
\begin{center}
    \zihao{5}
    本文使用\LaTeX 进行写作
\end{center}
\end{document}